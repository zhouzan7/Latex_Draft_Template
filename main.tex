\documentclass{article}
\usepackage{ctex}  % 自动支持中文
\usepackage{geometry}
\geometry{a4paper, margin=1in}

\usepackage{graphicx}  
\usepackage{booktabs}    
\usepackage{cite}

\bibliographystyle{plain}

\begin{document}

\title{示例文档}
\author{你的名字}
\date{\today}
\maketitle

\section{引言}
这是一个支持中文和 BibTeX 的 LaTeX 示例文档模板。 

\section{方法}
这里是方法部分。

\subsection{插入表格}
你可以插入表格,如表\ref{tab:feature_map_sizes}所示:
  \begin{table}[hbt]
    \centering
    \caption{MobileNetV4-Conv-L特征层的尺寸和通道数} 
    \label{tab:feature_map_sizes}
    \begin{tabular*}{0.75\textwidth}{@{\extracolsep{\fill}}ccc}
    \toprule
      特征层 & 尺寸 & 通道数 \\
    \midrule
      \(\times2\)  & \(H/2 \times W/2\)   & 24 \\
      \(\times4\)  & \(H/4 \times W/4\)   & 48 \\
      \(\times8\)  & \(H/8 \times W/8\)   & 96 \\
      \(\times16\) & \(H/16 \times W/16\) & 192 \\
      \(\times32\) & \(H/32 \times W/32\) & 960 \\
    \bottomrule
    \end{tabular*}
  \end{table}

\subsection{插入图片}
你也可以插入图片,如图\ref{fig:seg}所示:
  \begin{figure}[hbt]
      \centering
      \includegraphics[width=0.45\textwidth]{figures/seg.png}
      \caption{语义分割结果图}\label{fig:seg}
     \end{figure}

\subsection{引用文献}
你可以引用文献,例如\cite{zhao2024autonomous}。

\section{结论}
这是结论部分。

\bibliography{main}

\end{document}
